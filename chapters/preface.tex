\chapter*{پیشگفتار}

    دانشجویان تحصيلات تکمیلی برای ارائه پایان‌نامه/رساله خود ملزم به رعایت چارچوب کلی تعیین شده توسط معاونت پژوهشی موسسه/دانشگاه مطبوع خود 
    هستند. با توجه به اینکه رعایت دقیق این نکات توسط دانشجو امری زمان‌بر بوده و در نهایت هم مستلزم بررسی توسط ناظر شکلی تحصیلات تکمیلی و 
    کتابخانه دانشگاه است، عموماً با توجه به حجم کار و گستردگی آن مستندات تحویلی یک دست نبوده و دقیقاً مطابق با آنچه در قانون آمده است 
    نخواهد شد و مسئولین امر برای اینکه دانشجو به مشقت نیفتند معمولاً با دیده اغماض به این اشکالات نگریسته و از آن در می‌گذرند. 
    به همین سبب در برخی مؤسسات اقدام به آماده‌سازی قالبی از پیش‌آماده می‌نمایند تا به میزان زیادی از این اشکالات ناخواسته جلوگیری گردد. 
    
    هر چند که امروزه نرم‌افزار مایکروسافت ورد انتخاب اول کاربران برای حروفچینی اسناد است لکن این نرم‌افزار یک حروفچین نبوده و تنها یک ویرایشگر 
    پیشرفته متن است. نکتهٔ فوق و دیگر اینکه دانشجویان علوم پایه و بعضاً فنی مهندسی بخصوص رشته‌های ریاضی، فیزیک، برق و کامپیوتر در اسناد 
    خود با فرمول‌های ریاضی سر و کار دارند بهترین انتخاب را سیستم حروفچینی لاتک \lr{(\LaTeX{})} می‌یابند --
    گرچه در گروه ریاضی و فیزیک دانشگاه قم دانشجویان ملزم به آماده‌سازی پایان‌نامه خود با لاتک هستند--. 
    دانشجویان با وجود لاتک و یک قالب آماده، 
     دیگر هیچ نگرانی برای حروفچینی متن و رعایت دستورالعمل نگارشی دانشگاه ندارند و تمامی موارد 
    --همچون اندازه و نوع قلم متن و عناوین، اندازه حاشیه‌ها، صفحات 
    آغازین، سبک منابع و مآخذ و \ldots \hspace{2mm}-- به صورت خودکار توسط قالب آماده شده اعمال می‌گردد. 
    از این نقطه به بعد دانشجویان، دیگر تنها کافی است که روی متحوای کار خود تمرکز نمایند. 
    اگرچه ممکن است برای برخی دانشجویان یادگیری دستورات لاتک در بدو امر کمی مشکل باشد، امّا به تدریج با دستورات آن آشنا خواهند شد و 
    در ادامه در خواهند یافت که چقدر حروفچینی با لاتک آسان و دلنشین است. 
    
    کلاس پایان‌نامه/رساله دانشگاه قم سعی نموده با نگاهی به تمامی کلاس‌های موجود، کلاسی را فراهم آورد که کار کردن با آن برای دانشجویان بسیار ساده باشد و به نظر 
    نیز چنین است. در این کلاس هیچ فیلد اجباری وجود ندارد و تمامی مقادیر به صورت پیش‌فرض مقداردهی می‌شوند و در صورتی که کاربر 
    مقداری برای فیلدهای متناظر تعریف نماید از آن فیلد‌ها استفاده خواهد شد. از جمله دیگر مزایای این کلاس، تمرکز اصلی دانشجو بر محتوای سند 
    است و لازم نیست که دستورات ویژه یا نکات خاصی را در نگارش خود رعایت نماید و کلاس سعی نموده است که تمامی کارهای لازمه را به صورت 
    خودکار انجام دهد. 
    
    قطعاً این قالب بدون نقص نبوده و در صورت دریافت بازخورد از سمت کاربران، توسعه‌دهندگان خود را متعهد به اصلاح آن می‌دانند. ضمناً در صورت 
    نیازهای جدید کاربران نیز تا آنجایی که معقول باشد بر خود وظیفه می‌دانند که آن‌ها را نیز بمرور زمان و در حد امکان برآورده نمایند. امید است 
    این قالب وظیفه دانشجویان را در آماده‌سازی پایان‌نامه/رساله تسهیل نماید و ذهن آنان را معطوف به متن اصلی خود نماید. 
