% !TEX TS-program = XeLaTeX
% !TeX root=main.tex

\chapter{آشنایی سریع با برخی دستورات لاتک}\label{Chap:chapter2}

در این فصل ویژگی‌های مهم و پرکاربرد زی‌پرشین و لاتک معرفی می‌شود. برای راهنمایی بیشتر و به کاربردن ویژگی‌های پیشرفته‌تر به راهنمای زی‌پرشین 
و راهنمای لاتک مراجعه کنید. 

\section{بندها و زیرنویس ها}
هر جایی از نوشتهٔ خود، اگر می خواهید به سر سطر بروید و یک پاراگراف تازه را آغاز کنید، باید یک خط را خالی بگذارید.

حالا که یک بند تازه آغاز شده است، یک زیرنویس انگلیسی%
\LTRfootnote{English Footnote!}
 هم می نویسیم!
\section{فرمول‌های ریاضی}\label{formula}

اینجا هم یک فرمول می آوریم که شماره دارد:
\begin{equation}\label{eq:yek}
A=\frac{c}{d}+\frac{q^2}{\sin(\omega t)+\Omega_{12}}
\end{equation}

%\RTLcolumnfootnotes

در لاتک می توان به کمک فرمان 
\Verb+\label{}+
به هر فرمول یک نام نسبت داد. در فرمول بالا نام \lr{eq:yek} را برایش گذاشته‌ایم (پروندهٔ \lr{tex} همراه با این مثال را ببینید). این نام ما را قادر می‌کند که بعداً بتوانیم با فرمان
\Verb+\ref{eq:yek}+
به آن فرمول با شماره ارجاع دهیم. یعنی بنویسیم فرمول \ref{eq:yek}. 
لاتک خودش شمارهٔ این فرمول‌ها را مدیریت می‌کند. یعنی اگر بعداً فرمولی قبل از این فرمول بنویسیم، خود به خود شمارهٔ این فرمول و شمارهٔ ارجاع‌ها 
به‌ این فرمول یکی زیاد می‌شود و لازم نیست نگران شماره گذاری فرمول‌های خود باشید.

این هم یک فرمول که شماره ندارد:
$$A=|\vec{a}\times \vec{b}| + \sum_{n=0}^\infty C_{ij}$$

این هم عبارتی ریاضی مانند 
$\sqrt{a^2+b^2}$
 که بین متن می آید.

نمایش ارقام در محیط‌های مختلف متفاوت است. به عنوان مثال اگر \lr{0123456789.123} را  در حالت متن و ریاضی فارسی و در حالت معمولی و پررنگ لاتین داشته باشید، خروجی به ترتیب به صورت زیر خواهد بود:
 \begin{LTR}
 \noindent
 0123456789.123\\
 $0123456789.123$\\
 \lr{0123456789.123}\\
\lr{ $\mathbf{0123456789.123}$}\\
\end{LTR}
 در حالت کلی ارقام در حالت متن فارسی از قلم فارسی و در متن انگلیسی از قلم انگلیسی گرفته می‌شوند. 
 برای تغییر نوع و اندازه قلم ارقام در محیط ریاضی باید  دستور \Verb+\setdigitfont+ را بکار برد. 
 هر چند که در قالب  \LR{\Verb+thesis-qom+} این تنظیمات مطابق با استاندارد دانشگاه به صورت خوکار انجام گرفته و جای هیچگونه نگرانی وجود ندارد. 
 در این قالب تمامی ارقام در حالت متن پارسی نیز با همان قلمی نگاشته می‌شوند که متن ریاضی حروفچینی می‌شود و همین سبب یکدست شدن 
 ارقام متن می‌گردد. 
 
   ممکن است خواسته باشید برخی ارقام ریاضی را - مثلاً برای نمایش یک بردار - با حروفی متفاوت نشان دهید، مثل این: 
\begin{LTR}
 \noindent
$\mathsf{0123456789.123}$ 
\end{LTR}


که از دستور 
\Verb!\mathsf{0123456789}!
برای نمایش آن استفاده شده است. برای این مثال از قلم \lr{IRTitr}
در دستور \Verb!\setmathsfdigitfont{IRTitr}!
استفاده شده است.

\subsection{یک زیربخش}\label{zirbakhsh}

این زیربخش \ref{zirbakhsh} است؛ یعنی یک بخش درون بخش \ref{formula} است.
\subsubsection{یک زیرزیربخش}
این هم یک زیرزیربخش است. در لاتک می‌توانید بخش‌های تو در تو در نوشته‌تان تعریف کنید تا ساختار منطقی نوشته را به خوبی نشان دهید. 
می‌توانید به‌ این بخش‌ها هم با شماره ارجاع دهید، مثلاً بخش فرمول‌های ریاضی شماره‌اش \ref{formula} است.
\section{نوشته‌های فارسی و انگلیسی مخلوط}
نوشتن یک کلمهٔ انگلیسی بین متن فارسی بدیهی است، مانند Example در این جمله.
نوشتن یک عبارت چندکلمه‌ای مانند
More than one word کمی پیچیده‌تر است.
    همانطور که متوجه شده‌اید جمله قبل به صورت راست به چپ حروفچینی شده است؛ به طور کلی برای حروفچینی یک کلمه/جمله انگلیسی در متن فارسی
    همیشه دستور \Verb+\lr{}+ را به کار برید تا هم برای حروفچینی از قلم لاتین استفاده شود و هم اینکه از چپ به راست چیده شود: \lr{More than one word}.
    به تفاوت این جمله اخیر با آن دو عبارت قبلی لاتین خوب توجه نمایید. 

اگر ناگهان تصمیم بگیرید که یک بند کاملاً انگلیسی را بنویسید، باید آن را درون محیط \Verb+latin+ قرار دهید:‌

\begin{latin}
This is an English paragraph from left to right. You can write as much as you want in it.
\end{latin}

    بالعکس اگر بخواهید درون متن لاتین کلمات فارسی داشته باشید باید از دستور \Verb+\rl{}+ و یا محیط \Verb+persian+ استفاده نمایید. 
    
\section{افزودن تصویر به نوشته}
پروندهٔ تصویر دلخواه خود را در کنار پروندهٔ \lr{tex} قرار دهید. سپس به روش زیر تصویر را در نوشتهٔ خود بیاورید --توجه نمایید که هیچ نیازی به درج پسوند 
فایل تصویر وجود ندارد و بستهٔ \Verb+graphicx+ به صورت خودکار این کار را انجام خواهد داد--:
%\begin{latin}
\begin{Verbatim}
\includegraphics{YourImageFileName}
\end{Verbatim}
%\end{latin}
     اکیداً پیشنهاد می‌شود که تصاویر در یک پوشه مجزا برای نمونه تحت نام \Verb+images+ قرار دهید و سپس به صورت 
    \Verb+\includegraphics{images/YourImageFileName}+\footnote{
    اگر کاربر ویندوز هستید ممکن است این اشکال به نظرتان آید که جداکننده آدرس در این سیستم‌عامل \textbackslash{}     است و ما باید در دستور درج تصاویر 
    آدرس را با این سمبل جدا می‌نمودیم. لکن باید توجه داشته باشید که دستورات در لاتک با \textbackslash{}  شروع می‌شود لذا برای جداکننده 
    آدرس یا باید /  را بکار برده و یا از \textbackslash\textbackslash استفاده کرد.}
    تصاویر را فراخوانید. از آنجایی که درج اسامی تصاویر به همراه آدرس پوشه‌ای که درون آن قرار دارند ممکن است کمی زمان‌بر به نظر آید لذا برای رهایی از آن می‌توان 
    مسیر/مسیرهایی را که تصاویر در آن قرار دارند به لاتک معرفی کنیم تا به صورت خودکار تصاویر بدون نیاز به ذکر آدرس دقیق آن‌ها، از همان آدرس درج گردند. دستور ذیل 
    این کار را میسر می‌نماید:
\begin{Verbatim}
\graphicspath{{PATH1}{PATH2}{PATH3}...}
\end{Verbatim}    
    به تصویرها هم مانند فرمول‌ها و بخش‌ها می توان با شماره ارجاع داد. برای جزئیات بیشتر دربارهٔ روش گذاشتن تصویرها در نوشته باید راهنماهای لاتک 
    را بخوانید. نمونه تصاویری در پیوست آمده است که می‌توانید نحوه درج آن‌ها را ملاحظه فرمایید.


\section{محیط‌های شمارش و نکات}
برای فهرست کردن چندمورد، اگر ترتیب برایمان مهم نباشد:
\begin{itemize}
    \item مورد یکم
    \item مورد دوم
    \item مورد سوم
\end{itemize}
و اگر ترتیب برایمان مهم باشد:
\begin{enumerate}
    \item مورد یکم
    \item مورد دوم
    \item مورد سوم
\end{enumerate}
می‌توان موردهای تو در تو داشت:
\begin{enumerate}
    \item مورد ۱
    \item مورد ۲
        \begin{enumerate}
            \item مورد ۱ از ۲
            \item مورد ۲ از ۲
            \item مورد ۳ از ۲
        \end{enumerate}
    \item مورد ۳
\end{enumerate}
شماره‌گذاری این موارد را هم لاتک انجام می دهد؛ البته این امکان وجود دارد که نوع شماره‌گذاری را تغییر دهید. 

\section{تعریف و قضیه}

برای ذکر تعریف، قضیه و مثال مثالهای ذیل را ببینید.

%\textbf{برای ذکر تعریف، قضیه و مثال مثالهای ذیل را ببینید.}
%
%{\iranicfamily برای ذکر تعریف، قضیه و مثال مثالهای ذیل را ببینید.}
%
%\textbf{\emph{ برای ذکر تعریف، قضیه و مثال مثالهای ذیل را ببینید.}}
%
%{\iranicfamily \bf{ برای ذکر تعریف، قضیه و مثال مثالهای ذیل را ببینید.}}

\begin{definition}
مجموعه همه ارزیابی‌های  (پیوسته)  روی $(X,\tau)$، دامنه توانی احتمالی
\index{دامنه توانی احتمالی}
$ X $
نامیده می‌شود.
\end{definition}

\begin{theorem}[باناخ-آلااغلو]
\index{قضیه باناخ-آلااغلو}
اگر $ V $ یک همسایگی $ 0 $ در فضای برداری 
\index{فضای!برداری}
 توپولوژیکی $ X $ باشد و 
\begin{equation}\label{eq1}
K=\left\lbrace \Lambda \in X^{*}:|\Lambda x|\leqslant 1 ; \ \forall x\in V\right\rbrace,
\end{equation}
آنگاه $ K $،  ضعیف*-فشرده است که در آن، $ X^{*} $ دوگان \index{فضای!دوگان}
 فضای برداری توپولوژیکی $ X $ است به  طوری که عناصر آن،  تابعی های  خطی پیوسته\index{تابعی خطی پیوسته}  روی $X$ هستند.
\end{theorem}

تساوی \eqref{eq1} یکی از مهم‌‌ترین تساوی‌ها در آنالیز تابعی است که در ادامه، به وفور از آن استفاده می‌شود.

\begin{example}
برای هر فضای مرتب، گردایه 
$$U:=\left\lbrace U\in O: U=\uparrow U\right\rbrace $$
از مجموعه‌های بالایی باز، یک توپولوژی تعریف می‌کند که از توپولوژی اصلی، درشت‌تر  است.
\end{example}
حال تساوی 
\begin{equation}\label{eq2}
\sum_{n=1}^{+\infty} 3^{n}x+7x=\int_{1}^{n}8nx+\exp{(2nx)}
\end{equation}
را در نظر بگیرید. با مقایسه تساوی \eqref{eq2} با تساوی \eqref{eq1} می توان نتیجه گرفت که ...


\section{چگونگی نوشتن و ارجاع به مراجع}\label{Sec:Ref}

در لاتک به راحتی می‌توان مراجع خود را نوشت و به آن‌ها ارجاع داد. به عنوان مثال برای معرفی کتاب گنزالس \cite{Gonzalez02book} به عنوان یک مرجع 
می‌توان آن را به صورت زیر معرفی نمود:
\begin{latin}
\begin{Verbatim}
\bibitem{Gonzalez02book}
Gonzalez, R.C., and Woods, R.E. {\em Digital Image Processing}, 3rd ed.,
Prentice-Hall, Inc., Upper Saddle River, NJ, USA, 2006.
\end{Verbatim}
\end{latin}

در دستورات فوق \lr{Gonzalez02book}  برچسبی است که به‌این مرجع داده شده است و با استفاده از دستور \Verb!\cite{Gonzalez02book}!
می توان به آن ارجاع داد؛ بدون این که شماره‌اش را در فهرست مراجع بدانیم.

اگر این اولین مرجع ما باشد در قسمت مراجع به صورت زیر خواهد آمد:\\
\begin{latin}
\begin{enumerate}
    \item [{[1]}] Gonzalez, R.C., and Woods, R.E. {\em Digital Image Processing}, 3rd ed.,
    Prentice-Hall, Inc., Upper Saddle River, NJ, USA, 2006.
\end{enumerate}
\end{latin}

این شیوه برای تعداد مراجع کم بد نیست امّا اگر فرمت مراجع، ترتیب یا تعداد آن‌ها را خواسته باشید تغییر دهید، به عنوان مثال ابتدا حرف اول نام نویسنده بیاید 
و سپس نام~خانوادگی، باید همه کارها را به صورت دستی انجام دهید.
%اگر مایلید کنترل کاملی بر مراجع خود داشته باشید و به راحتی بتوانید قالب مراجع خود را عوض کنید باید از \lr{Bib\TeX} استفاده کنید که درپیوست  \ref{App:App1} به  آن پرداخته خواهد شد. 

     همیشه یکی از بخش‌های چالشی برای دانشجویان و پر از اشکال برای ناظر شکلی، بخش مراجع پایان‌نامه/رساله است که متاسفانه دانشجویان آن را به 
     درستی رعایت نمی‌نمایند و هر مدخل از یک سبک استفاده نموده و هماهنگی بین آن‌ها وجود ندارد. برای جلوگیری از این  رخداد، 
     در قالب \LR{\Verb+thesis-qom+} به شما امکان استفاده از این شیوه برای نگارش مراجع داده نمی‌شود تا از اینگونه اشکالات جلوگیری شود. 
     در عوض باید تمامی مراجع مطابق با سبک نگارش \lr{Bib\TeX}  در فایل \Verb+references.bib+ درج گردد و مابقی کارها را به \lr{Bib\TeX} سپرد. 
     با این شیوه در صورتیکه که قرار باشد سبک حروفچینی مراجع نیز تغییر نمایید بدون هیچ زحمتی و تنها با افزودن سبک مورد نظر می‌توان به مقصود رسید. 
     
