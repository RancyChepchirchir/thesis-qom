    \chapter{راهنمای نصب  \lr{\LaTeX}} 
    \label{chap:installation}

    \section{مقدمه}
    نرم‌افزار حروفچینی \lr{\TeX} یکی از نرم‌افزارهای معروف حروفچینی متون علمی است
    که در سطح وسیعی جهت حروفچینی  مجلات و کتب استفاده می‌شود. در این متن مختصر بر آنیم
    که راهنمای سریعی برای نصب و استفاده از آن بیان کنیم با این امید که کاربران با پیگیری آن
    به راحتی  بتوانند آن را نصب و استفاده نمایند.

    قبل از این لازم است جهت واضح شدن شکل عملکرد این نرم افزار، اطلاعاتی در مورد آن داشته
    باشیم که در ادامه به آن پرداخته می‌شود.

    نرم افزار حروفچینی \lr{\TeX} یک نرم افزار مجانی است که به صورت خط فرمانی کار می‌کند، به این
    معنی که متن مورد نظر در یک فایل نوشته شده و سپس این فایل از طریق دستورات خط
    فرمان به نرم افزار حروفچین \lr{\TeX} داده می‌شود. این نرم افزار فایل داده شده را خوانده و بر
    مبنای آن متن حروفچینی شده را به صورت یک فایل (مثلا \lr{PDF}) ارائه می‌کند.

    ابزارهای پردازشی خط فرمان متعددی برای استفاده از این نرم افزار حروفچین وجود دارد که از مهمترین 
    آن‌ها می‌توان به \lr{latex}، \lr{pdflatex} و \lr{xelatex} اشاره کرد. معمولاً ما این بخش از
    نرم افزار حروفچین را موتور \lr{\TeX} می‌نامیم. این خاصیت، اولین متمایز کنندۀ این نرم‌افزار
    از سایر نرم‌افزارها نظیر \lr{Office}  است زیرا در \lr{Office} شما نتیجه نهایی را همزمان با تایپ
     می‌بینید ولی در این نرم‌افزار باید فایل را به حروفچین بدهید تا خودش شکل خروجی را آماده
     کند. عملاً به همین دلیل نیز آن را نرم‌افزار حروفچین می‌نامند، مشابه این که شما متن خام 
     خود را به یک فرد حروفچین می‌دهید تا با شکل دهی آن در قالب صفحات، آن را برای چاپ
     آماده کند.
     
     پس متن خام باید در یک ویرایشگر تایپ شده و سپس فایل حاصل (که پسوند آن \lr{.tex } است)
     به برنامۀ حروفچین
     با استفاده از خط فرمان داده شود. ویرایشگرهایی وجود دارند که امکان وارد کردن متن خام
     و به طور همزمان، امکان دادن فایل به موتور \lr{\TeX} و نشان دادن نتیجۀ حروفچینی را دارند. 
     اما تمام آن‌ها بر مبنای همان دستورات خط فرمان عمل می‌کنند و هیچکدام به تنهایی و بدون
     دسترسی به یک موتور \lr{\TeX} نمی‌توانند خروجی تولید کنند. البته هیچ وابستگی بین
     ویرایشگر و فایل تولید شده توسط آن وجود ندارد و یک فایل توسط هر کدام می‌تواند 
     تولید یا ویرایش شود یا فایل ایجاد شده توسط  یک ویرایشگر، در دیگری تغییر یابد.
     از معروف‌ترین این ویرایشگرها می‌توان به 
     \lr{WinEdit}، \lr{Texmaker}، \lr{TeXstudio}
    و  \lr{Notepad++}  اشاره کرد--اولی و آخری تنها برای سیستم عامل ویندوز موجودند. از جملهٔ ویرایشگرهایی که در دوره‌ای  میان کاربران پارسی 
    عمومیت یافت، \lr{bidiTeXmaker} بود که توسط آقای سیدرضی علوی‌زاده با افزودن مشخصه‌هایی برای کاربران پارسی‌زبان، توسعه داده شد\cite{biditexmaker}.

    \section{نصب موتور اصلی \lr{\TeX}}
    توزیع‌های مختلفی برای موتور \lr{\TeX} وجود دارد که در اینجا به نصب دو توزیع
    معروف و مجانی آن به نام‌های \lr{\TeX{}Live}  و \lr{Mik\TeX} می‌پردازیم. تاکید می‌شود که این توزیع‌ها با هم سازگار هستند، به این
    معنی که فایل آماده شده روی تمام توزیع‌های موتور \lr{\TeX} \ کار می‌کند. لذا که مهم نیست کدام
    توزیع را برای نصب انتخاب کنید.
    بسته \lr{\XePersian}  نصب می‌شود و نیاز به هیچ کار اضافی نیست. فقط لازم است
    که فونت‌های فارسی استفاده شده در متون فارسی روی سیستم~عامل نصب شده باشد. لذا
    تنها کار اضافی این است که مجموعه فونت‌های جمع آوری شده در فایل زیر روی سیستم عامل
    نصب شود. توصیه می‌شود حتی اگر فونت‌ها را روی کامپیوتر خود دارید، دوباره آن‌ها را با استفاده
    از فونت‌های فایل زیر رونویسی کنید. این کار از بسیاری مشکلات بعدی جلوگیری می‌کند.
    \begin{latin}

    \href{http://bayanbox.ir/id/4609192605141061595}{Part 1: http://bayanbox.ir/id/4609192605141061595}

    \href{http://bayanbox.ir/id/5468937351173971771}{Part 2: http://bayanbox.ir/id/5468937351173971771}

    \href{http://bayanbox.ir/id/4133277893427051503}{Part 3: http://bayanbox.ir/id/4133277893427051503}
    \end{latin}

    البته توصیه اکید پدیدآورنده بسته \lr{\XePersian} جناب دکتر وفا خلیقی  که جهت تولید متون فارسی در \lr{\TeX} 
    این بسته را ارائه کرده‌اند، استفاده از  \lr{\TeX{}Live}  است. 
    %\pagebreak
    \subsection{نصب \lr{\TeX{}Live}}
    سایت‌های معروف به \lr{CTAN}، سایت‌هایی هستند که  وظیفه توزیع نسخه‌های مختلف
    مجانی موتور \lr{\TeX} را انجام می‌دهند. با توجه به اینکه معمولاً سرعت دانلود از سایت‌های داخلی 
    بیشتر بوده و اخیرا نیز هزینه دانلود از این سایت‌ها به صورت نیم‌بها محاسبه می‌گردد لذا توصیه می‌شود به یکی از سه 
    سایتی که در ایران وجود دارد مراجعه نموده و توزیع تک‌لایو را دانلود نمایید:
    \begin{enumerate}
        \begin{LTRitems}
            \item \url{http://ctan.asis.io/}
            \item \url{http://ctan.yazd.ac.ir/}
            \item \url{http://repo.iut.ac.ir/tex-archive/}
        \end{LTRitems}
    \end{enumerate}
        امید است دیگر دانشگاه‌های ایران نیز مانند دانشگاه یزد و اصفهان اقدام به ایجاد یکی از این سایت‌ها روی سرورهای خود نمایند تا دانشجویان آن موسسات 
        بتوانند براحتی و بدون از دست دادن حجم اکانتینگ خود به مجموعه آرشیو تک دسترسی داشته باشند. 

    این سایت به صورت روزانه به روز رسانی می‌شود. می‌توان از این سایت در هر 
    لحظه آخرین نگارش‌های نرم افزارهای مربوطه را دانلود کرد. 

    برای نصب \lr{\TeX{}Live}  مراحل زیر را انجام دهید:
    \begin{enumerate}
    \item       
        ابتدا وارد یکی از سایت‌های که جلوتر اشاره شد 
     شوید و در پایین صفحه روی \lr{\TeX{}Live}
     کلیک کنید.
     \item
      روی مسیر \lr{Images} کلیک کنید و از فولدر باز شده فایل با نام 
     	\lr{texlive.iso} 
     	را دانلود کنید. دقت کنید که حجم این فایل  در حال حاضر 
	    حدود $3.4$ گیگابایت است.
     \item
      پس از دانلود کامل، آن را با نرم افزار \lr{WinRaR}  باز کنید و در  پوشه‌ای به نام \lr{TeXLive} فایل را \lr{Extract} کنید.
     \item 
     وارد این پوشه شوید و برنامه \LR{\Verb+install-tl-windows+}
     را اجرا کنید. ادامه روند مشابه نصب سایر نرم افزارها 
     است. روند نصب بسته به سرعت کامپیوتر شما ممکن است تا یک ساعت طول بکشد.

     \item
      پس از پایان نصب، موتور \lr{\TeX} آماده استفاده است. اگر قصد استفاده از \lr{\XePersian}
     \ دارید، فقط لازم است فونت‌های مربوطه را که در بالا لینک آن آمده است را نصب کنید.
    \end{enumerate}
    بهتر است بعد از نصب؛ بسته‌های این نرم افزار را با روش زیر به روز رسانی کنید.
    \subsubsection{بروزرسانی بسته‌های \lr{\TeX{}Live} }
    دقت کنید که برای بروزرسانی شما باید به اینترنت متصل باشید زیرا بروزرسانی با استفاده از اینترنت انجام می‌شود.
    \begin{enumerate}
    \item ابتدا در قسمت برنامه‌ها، برنامه \lr{\TeX{}Live manager} را اجرا کنید.
    \item 
        مسیر به روزرسانی را یکی از سایت‌های داخلی انتخاب کنید. انتخاب هر مسیر دیگر اشکالی ندارد ولی روی سرعت گرفتن فایل‌ها و هزینه اینترنت تاثیر مستقیم دارد. 
    \item 
        سپس بسته‌های مشخص شده را به روزرسانی کنید. پس از بروز رسانی این بسته‌ها، برنامه بسته می‌شود و لازم است دو مرحله قبل تکرار شود.
        البته با این روش می‌توانید تنها بسته خاصی را بروزرسانی نمایید لکن این حالت خیلی توصیه نمی‌شود زیرا بسیاری از بسته‌ها به یکدیگر وابسته هستند. 
    \item 
        حال روی \lr{Updtate all installed} کلیک کنید. به روزرسانی نیز مشابه نصب مدت زمانی که به سرعت کامپیوتر و سرعت اینترنت شما وابسته است طول می‌کشد.
    \end{enumerate}

    \subsection{نصب  \lr{Mik\TeX}}
        از آنجایی که توصیه اکید توسعه‌دهندگان زی‌پرشین بر استفاده از تک‌لایو است لذا این بخش خیلی توضیح داده نمی‌شود و تنها به همین میزان 
        اکتفا می‌گردد که می‌توانید از همان سایت‌هایی که پیشتر معرفی گردیدند میک‌تک را دانلود نمایید. تنها نکته‌ای که باید توجه داشته باشید این است 
        که میک‌تک در نسخه‌ مینیمال نیز عرضه می‌گردد لکن برای استفاده از زی‌پرشین این نسخه‌ها ناکارآمد است و باید نسخه کامل آن را نصب نمایید. 
        

    \section{نصب \lr{Notepad++}}
    ادیتور \lr{Notepad++}  به دلیل قابلیت فارسی نویسی و همچنین از راست به چپ نویسی و امکان اجرای دستورات خط فرمان در ادیتور،
    انتخاب مناسبی برای نوشتن متون  است. برای فعال کردن قابلیت اجرای دستورات خط فرمان با استفاده از کلید \lr{F6}، پس از نصب نرم افزار \lr{Notepad++ }، 
    لازم است تا پلاگین \lr{NppExec} را نصب نمایید. بدین منظور از منوی     \lr{Plugins -> Plugin Manager -> Show Plugin Manager}
    پلاگین \lr{NppExec}  را انتخاب نموده و \lr{Install} را بزنید تا پلاگین مورد نظر نصب شود. البته این ادیتور پلاگین‌های بسیار زیادی دارد 
    که قابلیت‌های خوبی را به آن می‌افزاید که می‌تواند به کمکتان آید لذا بررسی آن‌ها خالی از فایده برایتان نخواهد بود.
    اگر از این طریق قادر به نصب پلاگین مورد نظر نشدید می‌توانید با مراجعه به آدرس \url{https://sourceforge.net/projects/npp-plugins/files/NppExec/}
    آن را دانلود نموده و سپس محتویات فایل زیپ را  در پوشه \lr{Plugins} که در محل نصب \lr{Notepad++}  قرار دارد کپی نمایید. 
    حال با زدن کلید \lr{F6}  در ادیتور، پنجره اجرای دستور باز می‌شود.
    نمونه دستوری که می‌توانید وارد کنید به صورت زیر است:

\begin{latin}
    \begin{Verbatim}
    NPP_SAVE
    cd $(CURRENT_DIRECTORY)
    xelatex --shell-escape $(NAME_PART)
    \end{Verbatim}
\end{latin}    

    برای تایپ از راست به چپ کلیدهای \lr{Alt+CTRL+R}  را بزنید و برای از چپ به راست نویسی کلیدهای \lr{Alt+Ctrl+L}  را بزنید.

    برای نیم فاصله، کلید استاندارد \lr{Ctrl+SHift+2}  است که در این ادیتور به دلیل استفاده از این ترکیب برای کار دیگری عمل نمی‌کند.
    برای عمل کردن آن باید این ترکیب کلید را از ادیتور حذف کنید. برای این منظور از منوی \lr{Settings -> Shortcut Mapper}
     در برگه \lr{Main Menu}  در ردیف حدودا ۱۱۰  این ترکیب را پیدا کرده و به چیز دیگری (مثلا \lr{CTRL+Shift+T}) عوض کنید.

     پس از این کار ترکیب  \lr{Ctrl+SHift+2} برای نیم فاصله (وقتی زبان فارسی باشد) کار می‌کند.
     
     توجه: برای تهیه فایل مقاله یا کتاب با \lr{\XePersian}، باید از کد \lr{UTF8}  برای کدگذاری فایل استفاده شود. برای انتخاب در ادیتور، از منوی
     \lr{Encoding}  گزینه مورد نظر انتخاب شود.
